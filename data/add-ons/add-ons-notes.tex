\documentclass[a4paper]{article}
% !TEX program = XeLaTeX+shell-escape
\input{preamble.tex}
\usepackage{hyperref}
\DeclareMathOperator{\Res}{Res}
\title{Add-ons methodology}
\begin{document}
  \maketitle
  \section{State Taxes} 
State taxes growth with nominal gdp. Let $s_i$ denote state taxes where $i = \text{personal,
corporate, production and imports, and social insurance}$, $y$ refer to nominal gdp growth,  and $t$ denote time. Then, the forecasted state taxes are
given by 
state taxes grow with nominal gdp. That is, 

\[
  s_{i,t}= s_{i,t-1} * y_{t}
.\]
\section{Grants}

\subsection{Medicaid NIPA's}
In the NIPA's, all medicaid spending is registered as state spending. The nipa definition of
medicaid spending in Haver is \textbf{yptmdx}. In the forecast period, we use the annual
CBO projections for Medicaid spending until Q3 of 2021.
\[
  \text{yptmdx}_{y, q} = \text{yptmdx}_{y, q-1}  * (\frac{CBO_{y}}{CBO_{y-1}})^{g}
.\] 
where the growth rate $g$ is  $0.18$ in Q4 of 2020 and Q1 of 2021 and  $0.1$ in Q2 of 2021.
\begin{align*}
  g = \begin{cases}
    0.18 & \text{in Q4 of 2020 and Q1 of 2021} \\
    0.10 & \text{in Q2 of 2021} \\
    0 & \text{ after Q2 of 2021 }
  \end{cases} 
.\end{align*}
Therefore, we assume medicaid spending is constant after Q2 of 2021.
\subsection{Federal Medicaid grants}
In Haver, federal medicaid grants are \textbf{gfeghdx} 
In the forecast period, we assume these grants are a specific share of the Medicaid spending in the
NIPA's. That is, 
\[
\text{gfeghdx}_t = \text{federal share} * \text{yptmdx}
.\] 
where 
\begin{align*}
  \text{Federal share} = \begin{cases}
    0.74 & \text{Q4 2020 } \le t \le \text{Q1 of 2022} \\
    0.68 & t > \text{Q1 of 2022}
  \end{cases}
.\end{align*}

\[
\text{gfeghdx}_t= 
.\] 
\subsection{All Non-medicaid grants}
Using haver codes, non Medicaid grants are
\[
\text{Non-medicaid grants} = gfegx - gfeghdx
.\] 
where \textbf{gfegx} denotes all grants to states and \textbf{gfeghdx} refers to Medicaid grants. The $x$ at the end
of these codes specifies that they are in millions of dollars. We \emph{only} define this for the
\textbf{historical period}

\subsection{BEA legislation grants}
Haver code is \textbf{gfegl}. 
\subsection{Non-medicaid grants excluding legislation}
In the historical period we simply define this to be the difference between BEA's grants legislation
 and all non-medicaid grants.

In the forecast period we grow non-medicaid grants excluding legislation by nominal federal
purchases (\textbf{gf}). Let $\mu$ be non-medicaid grants exluding legislation. Then, in the
forecast period we have 
 \[
   \mu_t = \mu_{t-1} * gf_{g, t}
.\] 
\subsection{Our estimate of legislation grants}
Our estimates of the legislation grants start in Q2 of 2020 and go through Q1 of 2023. They are
defined as 
\[
  \text{Grants} = \text{Grants to State and Local Govt's} + \text{Support for hospitals
  (state)}+\text{Education}
.\] 
Equivalently letting, $\lambda_{i}$ where  $i =  \text{ S\&L Govt's, Hospitals, and Education } $
we can rewrite this more compactly as 
\[
  \lambda^{*}_{t} = \sum_{i=1}^{3} \lambda_i 
.\] 
\remark
Currently, these values are hardcoded but I'm assuming they come from scores?
\subsection{All grants to states}

In haver, all grants to states is \textbf{gfeg} (in billions) and \textbf{gfegx}(in millions).
In the forecast period, all grants to states is equal Non medicaid grants excluding legislation plus the sum of our
estimate of the grants legislation from Q3 2020 through Q4 of 2022 divided by 6. In other words,
it's all non medicaid grants to states plus our estimate of the total grants in legislation for
COVID distributed equally over the six quarters. 

Let $\gamma$ be all grants to states, $\mu$ refer to non-medicaid grants exluding legislation and $\lambda^{*}$ refer to our estimate 
of the legislation. Then, in the forecast period 
\[
  \boxed{\gamma = \mu + \frac{1}{6} \sum_{t=1}^{6} \lambda^{*}_{t}}
.\] 

\remark
I think we can make this more explicit in this spreadsheet by creating a row that is the sum of all our legislation
estimate divided by 6 for 6 quarters. 

\subsection{Consumption grants}
We define consumption grants as the difference between all grants to states $\gamma$ and federal medicaid
grants (\textbf{gfeghdx}). Letting $\mu$ equal non-medicaid grants minus legislation, 
\[
\psi = \lambda^{*} + \delta
.\] 
where $\psi = \text{ consumption grants }$
\section{State purchases}
\subsection{Real}
Real state purchases (\textbf{gsz}) are from haver in the historical period.
In the forecast period, we use the growth rates from CBO
\subsection{Nominal}
Nominal state purchases(\textbf{gsx}) are from haver in the historical period. 
\remark
Currently, the projections are hardcoded. 
\section{State Social Benefits (FIM Definition)}
We treat state social benefits a bit differently than the NIPA's. Specifically, we reallocate UI
from federal to state and a portion of Medicaid from State spending to Federal spending.
Therefore, 

\[
  FIM_{\text{state s.b.}} = NIPA_{\text{state s.b.}} + \text{State UI}
  - \text{Federal medicaid grants}
.\] 
\section{FIM State Health Outlays}
Since we take out federal medicaid grants from state medicaid spending, state health outlays are 

\[
\text{State health} = \text{Medicaid in the NIPA's} - \text{Federal medicaid grants}
.\] 
\section{Federal social benefits}
Federal social benefits in the nipas is equal to medicare, total unemployment insurance (BEA UI,
Federal UI, Lost Wages Supplemental Payment Assistance program (WLA), and State UI), and the remainder of total social benefits. We make this
decomposition because the BEA did not include the Trump UI (WLA) in their UI category.  
\subsection{Medicare}
Our medicare forecast is based on CBO's annual projections, after making some adjustments to
translate these projections to the nipas. Since we need to translate annual projections to quarterly
ones our forecast formula for medicare is
\[
  M_{y, q} = M_{y, q-1}*(\frac{M'_{y}}{M'_{y-1}})
.\] 
where $y$ refers to the fiscal year, $q$ to the quarter, $M$ to medicare in the NIPA's and $M'$ to
medicare in the annual CBO projections.

\subsection{Unemployment Insurance}
The UI projections come from line items 22-25 from
\href{https://www.google.com/url?sa=t&rct=j&q=&esrc=s&source=web&cd=&cad=rja&uact=8&ved=2ahUKEwiKrurfj5zuAhWsGVkFHak3C7MQFjAEegQIAhAC&url=https%3A%2F%2Fwww.bea.gov%2Fsystem%2Ffiles%2F2020-07%2Feffects-of-selected-federal-pandemic-response-programs-on-personal-income-2020q2-advance.pdf&usg=AOvVaw0M-6ew-4iD7r1mb5f7WIFx}{Effects
of Selected Federal Pandemic Response Programs on Personal Income, July 2020}
\remark
We need more clarity on how these projections were constructed.
\subsection{Remainder}
In the historical period, the remainder is defined simply as the difference between the federal social benefits line item in
the NIPA's (found in Table 3.2) and our calculations for Medicare, Rebate checks, total Unemployment
Insurance, and PPP and provider relief in personal income.
\[
\epsilon=\text{Medicare} - \text{Rebate Checks} - UI - PPP
.\]
where $\epsilon$ refers to the remainder.
In the forecast period, we grow the remaindre $\epsilon$ by $-0.75 \%$ until  2021 Q3 and by $1.5\%$
until 2022 Q3, and then by  $1\%$ in Q4 of 2022
\remark
We need more clarity on where these growth rates are coming from. 
\subsection{PPP}
These numbers come from CBO scores.

\section{Subsidies}
Subsidies are defined as
\[
\text{Subsidies} = \text{PPP}+\text{Aviation}+\text{Employee retention}+\text{Paid sick leave}
.\] 
All these numbers come from CBO scores on the impact of stimulus legislation. They can be found in
Haver PPP(gfsubp), the sum of the other three subsidies (gsubl), aviation (gfsubg), employee
retention (gsube), and paid sick leave (gfsubk)

\section{Add factors}
Currently, the FIM code incorporates COVID legislation from CBO projections. However, CBO doesn't
reallocate spending in the same way we do (particularly for UI and Medicaid) and makes different
timing assumptions (all spending happens in a single quarter). Basically, CBO will say when states
will \emph{recieve} grants but in order to construct the FIM we need to know when the money
will be \emph{spent}. Therefore, we need to spread out the
legislation score and reallocate Federal Medicaid Grants to federal spending as well as Unemployment
legislation. To make these adjustments, the current
methodology is to run the code and compare the output of health outlays, social benefits, taxes,
subsidies, and consumption grants to the calculations described above present in the add factors
spreadsheet. Let $\alpha$ denote the add factor $x$ be the output from the code for these variables 
and $x^{*}$ be the variable in the spreadsheet which accounts for COVID legislation. Then, the add
factors are defined as
\[
\alpha = x^{*} - x
.\] 

We then go back into the code and add the add factor  $\alpha$ to each $x$. Therefore, we manually
put
 \[
x^{*} = x + \alpha
.\] 
in the code. 

\remark
We could either 
\begin{itemize}
  \item Read the variables $x^{*}$ in the spreadsheet directly into the code so that they override the
    original  $x$'s 
  \item Perform these calculations directly in the code.   
\end{itemize}

We could do the first one pretty easily but in the long term we should try to make the latter option
work. 


\end{document}
